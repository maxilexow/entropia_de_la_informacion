\chapter{Metodología y resultados}

Para comenzar, hay que transformar las imágenes a vectores 1-D, similar a lo que sería una serie de tiempo. Para ello se utilizó el algoritmo de Hilbert, el cual recorre la imagen de la siguiente manera:

\begin{figure}[H]
    \centering
    \includegraphics[width=0.6\textwidth]{figures/hilbert_curve.jpg}
    \caption{Curva de Hilbert obtenida para n=64.}
    \label{fig:hilbert}
\end{figure}

Se tomaron imágenes cuadradas de 64x64 píxeles, 

\begin{figure}[H]
    \centering

    \begin{subfigure}{0.25\textwidth}
        \includegraphics[width=\linewidth]{figures/data/oil_1.jpg}
    \end{subfigure}
    \begin{subfigure}{0.25\textwidth}
        \includegraphics[width=\linewidth]{figures/data/oil_2.jpg}
    \end{subfigure}
    \begin{subfigure}{0.25\textwidth}
        \includegraphics[width=\linewidth]{figures/data/oil_3.jpg}
    \end{subfigure}

    \vspace{1mm}
    {\small Petróleo}
    \vspace{3mm}

    \begin{subfigure}{0.25\textwidth}
        \includegraphics[width=\linewidth]{figures/data/land_1.jpg}
    \end{subfigure}
    \begin{subfigure}{0.25\textwidth}
        \includegraphics[width=\linewidth]{figures/data/land_2.jpg}
    \end{subfigure}
    \begin{subfigure}{0.25\textwidth}
        \includegraphics[width=\linewidth]{figures/data/land_3.jpg}
    \end{subfigure}

    \vspace{1mm}
    {\small Tierra}
    \vspace{3mm}
    
    \begin{subfigure}{0.25\textwidth}
        \includegraphics[width=\linewidth]{figures/data/ship_1.jpg}
    \end{subfigure}
    \begin{subfigure}{0.25\textwidth}
        \includegraphics[width=\linewidth]{figures/data/ship_2.jpg}
    \end{subfigure}
    \begin{subfigure}{0.25\textwidth}
        \includegraphics[width=\linewidth]{figures/data/ship_3.jpg}
    \end{subfigure}
    
    \vspace{1mm}
    {\small Barcos}

    \caption{Imágenes SAR utilizadas. Son recortes 64x64 de las imágenes originales, representando 3 clases.}
    \label{fig:mosaico_3x3}
\end{figure}

No se tomaron las clases look-alike ni océano, la primera por ser muy similar al petróleo, y la segunda por no tener prácticamente información, es el backgorund, y se ve como ruido.

\section{Patrones ordinales}
Se calcularon los patrones ordinales con $m=3,4,5$ y $tau=1,2$, obteniendo los siguientes resultados:

\begin{figure}[H]
    \centering
    \includegraphics[width=0.75\textwidth]{figures/patrones_ordinales/patrones_ordinales_m=3_tau=1.jpg}
    \caption{Distribución de patrones ordinales para m=3, tau=1.}
    \label{fig:hilbert}
\end{figure}


\begin{figure}[H]
    \centering
    \includegraphics[width=0.75\textwidth]{figures/patrones_ordinales/patrones_ordinales_m=3_tau=2.jpg}
    \caption{Distribución de patrones ordinales para m=3, tau=2.}
    \label{fig:hilbert}
\end{figure}


\begin{figure}[H]
    \centering
    \includegraphics[width=0.75\textwidth]{figures/patrones_ordinales/patrones_ordinales_m=4_tau=1.jpg}
    \caption{Distribución de patrones ordinales para m=4, tau=1.}
    \label{fig:hilbert}
\end{figure}


\begin{figure}[H]
    \centering
    \includegraphics[width=0.75\textwidth]{figures/patrones_ordinales/patrones_ordinales_m=4_tau=2.jpg}
    \caption{Distribución de patrones ordinales para m=4, tau=2.}
    \label{fig:hilbert}
\end{figure}


\begin{figure}[H]
    \centering
    \includegraphics[width=0.75\textwidth]{figures/patrones_ordinales/patrones_ordinales_m=5_tau=1.jpg}
    \caption{Distribución de patrones ordinales para m=5, tau=1.}
    \label{fig:hilbert}
\end{figure}


\begin{figure}[H]
    \centering
    \includegraphics[width=0.75\textwidth]{figures/patrones_ordinales/patrones_ordinales_m=5_tau=2.jpg}
    \caption{Distribución de patrones ordinales para m=5, tau=2.}
    \label{fig:hilbert}
\end{figure}

\section{Entropia y complejidad}
Luego se calcularon las entropias normalizadas de Shannon, Renyi y Tsallis, a la vez que las complejidades estadísticas de Jensen-Shannon, Jensen-Renyi y Jensen-Tsallis. Con todos estos datos se construyeron los planos complejidad-entropía y se plottearon puntos representando a cada imagen en dichos planos, junto a las curvas Cmín y Cmáx.

\begin{figure}[H]
    \centering

    \begin{subfigure}{0.32\textwidth}
        \includegraphics[width=\linewidth]{figures/hxc_plane/Shannon_HC_plane_m=3_tau=1.jpg}
    \end{subfigure}
    \begin{subfigure}{0.32\textwidth}
        \includegraphics[width=\linewidth]{figures/hxc_plane/renyi_alpha2_HC_plane_m=3_tau=1.jpg}
    \end{subfigure}
    \begin{subfigure}{0.32\textwidth}
        \includegraphics[width=\linewidth]{figures/hxc_plane/tsallis_q2_HC_plane_m=3_tau=1.jpg}
    \end{subfigure}

    \vspace{1mm}
    {\small m=3, tau=1}
    \vspace{3mm}

    \begin{subfigure}{0.32\textwidth}
        \includegraphics[width=\linewidth]{figures/hxc_plane/Shannon_HC_plane_m=4_tau=1.jpg}
    \end{subfigure}
    \begin{subfigure}{0.32\textwidth}
        \includegraphics[width=\linewidth]{figures/hxc_plane/renyi_alpha2_HC_plane_m=4_tau=1.jpg}
    \end{subfigure}
    \begin{subfigure}{0.32\textwidth}
        \includegraphics[width=\linewidth]{figures/hxc_plane/tsallis_q2_HC_plane_m=4_tau=1.jpg}
    \end{subfigure}

    \vspace{1mm}
    {\small m=4, tau=1}
    \vspace{3mm}
    
    \begin{subfigure}{0.32\textwidth}
        \includegraphics[width=\linewidth]{figures/hxc_plane/Shannon_HC_plane_m=5_tau=1.jpg}
    \end{subfigure}
    \begin{subfigure}{0.32\textwidth}
        \includegraphics[width=\linewidth]{figures/hxc_plane/renyi_alpha2_HC_plane_m=5_tau=1.jpg}
    \end{subfigure}
    \begin{subfigure}{0.32\textwidth}
        \includegraphics[width=\linewidth]{figures/hxc_plane/tsallis_q2_HC_plane_m=5_tau=1.jpg}
    \end{subfigure}
    
    \vspace{1mm}
    {\small m=5, tau=1}

    \caption{Planos complejidad entropía, m=1,2,3, tau=1.}
    \label{fig:planosHxC1}
\end{figure}

%%%%%%%%%%%%%%%%%%%%%%%%%%%%%%%%%%%%%%%%%%%%%%%%%

\begin{figure}[H]
    \centering

    \begin{subfigure}{0.32\textwidth}
        \includegraphics[width=\linewidth]{figures/hxc_plane/Shannon_HC_plane_m=3_tau=2.jpg}
    \end{subfigure}
    \begin{subfigure}{0.32\textwidth}
        \includegraphics[width=\linewidth]{figures/hxc_plane/renyi_alpha2_HC_plane_m=3_tau=2.jpg}
    \end{subfigure}
    \begin{subfigure}{0.32\textwidth}
        \includegraphics[width=\linewidth]{figures/hxc_plane/tsallis_q2_HC_plane_m=3_tau=2.jpg}
    \end{subfigure}

    \vspace{1mm}
    {\small m=3, tau=2}
    \vspace{3mm}

    \begin{subfigure}{0.32\textwidth}
        \includegraphics[width=\linewidth]{figures/hxc_plane/Shannon_HC_plane_m=4_tau=2.jpg}
    \end{subfigure}
    \begin{subfigure}{0.32\textwidth}
        \includegraphics[width=\linewidth]{figures/hxc_plane/renyi_alpha2_HC_plane_m=4_tau=2.jpg}
    \end{subfigure}
    \begin{subfigure}{0.32\textwidth}
        \includegraphics[width=\linewidth]{figures/hxc_plane/tsallis_q2_HC_plane_m=4_tau=2.jpg}
    \end{subfigure}

    \vspace{1mm}
    {\small m=4, tau=2}
    \vspace{3mm}
    
    \begin{subfigure}{0.32\textwidth}
        \includegraphics[width=\linewidth]{figures/hxc_plane/Shannon_HC_plane_m=5_tau=2.jpg}
    \end{subfigure}
    \begin{subfigure}{0.32\textwidth}
        \includegraphics[width=\linewidth]{figures/hxc_plane/renyi_alpha2_HC_plane_m=5_tau=2.jpg}
    \end{subfigure}
    \begin{subfigure}{0.32\textwidth}
        \includegraphics[width=\linewidth]{figures/hxc_plane/tsallis_q2_HC_plane_m=5_tau=2.jpg}
    \end{subfigure}
    
    \vspace{1mm}
    {\small m=5, tau=2}

    \caption{Planos complejidad entropía m=1,2,3, tau=2.}
    \label{fig:planosHxC2}
\end{figure}
