\chapter{Discusión}

Respecto a patrones ordinales...

Respecto a las figuras \ref{fig:planosHxC1} y \ref{fig:planosHxC2}, me parece que lo más importante a notar, es que es interesante trabajar con m=4 o m=5, y preferentemente con tau=1, ya que esos son los resultados en los que más separadas se ven las imágenes, y por lo tanto es más útil para separarlas en clases.

Como conclusión, podemos decir que, aunque no es el mejor método para trabajar con imágenes tipo SAR, puede tener aplicaciones interesantes para separar clases. Sus principales limitaciones son el tamaño de los crops, si son muy grandes mezclan varias clases en la misma imagen. La restricción de forma cuadrada también complica las cosas, sobre todo al tratar con features elongados, como los derrames de petróleo.

Como trabajo a futuro, es interesante seguir probando con muchas más imágenes y analizar más resultados para poder llegar a conclusiones más completas acerca de si el método es útil o no para el caso de la clasificación de imágenes SAR en océanos.